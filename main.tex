\documentclass[12pt, a4paper, openany]{report}
\usepackage[left=3cm, top=3cm, bottom=3cm, right=4cm]{geometry}
%\DeclareUnicodeCharacter{00A0}{}
\usepackage[numbers]{natbib}
\usepackage{secdot} % Dots in Section Numbers
\usepackage[utf8]{inputenc}
\usepackage[T1]{fontenc}
\usepackage{siunitx}
\sisetup{math-micro=\text{µ},text-micro=µ}
%\usepackage[sfdefault]{arimo}
\usepackage{amsmath}
\usepackage{amssymb}
\usepackage{graphicx}
\usepackage{color}% http://ctan.org/pkg/color
\usepackage{hyperref}
\usepackage{verbatim}
\usepackage{acronym}
\usepackage{graphicx}
\usepackage{csquotes}
\usepackage{setspace}
\usepackage{tabulary}
\usepackage[table]{xcolor}
\usepackage[leftcaption]{sidecap} % <-- added
\usepackage{booktabs}
\usepackage{notoccite}
\definecolor{light-gray}{gray}{0.75}
\hyphenation{di-me-thyl-tryp-ta-mine}
\hyphenation{an-ti-de-pres-sant}
\hyphenation{psy-che-del-ics}
\hyphenation{pa-ra-form-al-de-hyd}
\usepackage[textwidth=90, backgroundcolor=blue]{todonotes}
\hypersetup{
	colorlinks=true,
	urlbordercolor={1 1 1},
	linkcolor=black,
	raiselinks=false,
	urlcolor=black,
	filecolor=black,
	citecolor=black,
	anchorcolor=black
}
\usepackage{fancyhdr}
\usepackage{titlesec}
\titleformat{\chapter}[hang]{\Huge\bfseries}{}{0pt}{\Huge\bfseries}

\pagestyle{fancy}
\fancyhf{}
\lhead{ananym \& fux}
\chead{Everyone is broken inside}
\rhead{\thepage}

\title{
  {\large{Dostojewski Larp}}\\
    {Everyone is broken inside}\\
    {\bigskip}
}
\author{ananym \& fux}
\date{\today}


\begin{document}
\maketitle 
\tableofcontents
\chapter{Einleitung}

\chapter{Gruppen}
\section{Kellerlöcher}
\begin{itemize}
\item Rätsel in der sie Stück für Stück Beziehungen von sich entdecken 
\item Rätsel Lösung sind Auschnitte aus Charaktermap
\item dadurch entsteht ein 2. Rätsel: Charakteramap zusammen puzzeln 
\item dadurch kriegen sie Hinweis für Revolutionäre Gruppe Nastasja zu entführen
\item Jeder Briefumschlag Geld, Bild (Foto) und Rätsel 
\item Lösung von Rästel ist ein Codewort das sie durch die Tür schieben zurück kommen drei Umschläge mit jeweils einem Namen (angebliche freundschaftliche Beziehungen) und je unterschiedlich manipulierter Ausschnitt aus der Charaktermap
\item eine Person SL schiebt neue Info durch Türschlitzt nachdem sie Lösung durch Türschlitz schieben 
\end{itemize}
 
\section{Die Reichen}
\begin{itemize}
\item kennen sich nicht, aber in Charakterbeschreibung: es gibt eine Gruppe die an \textbf{euer} Geld will. 
\item Daher wollen sie rausfinden wer diese Gruppe ist und deren Ziel verhindern
\end{itemize}

\section{Die Revolutionäre Gruppe}
Ist am Anfang nicht auf der Party und weiß auch nichts von der Party (Einladung
für geheimes Orga Treffen)

Aufnahme Rituale für neue Mitglieder*innen (Kolja, Ipolit).

\begin{itemize}
\item Ziel 1: Geld verbrennen 
\item Ziel 2: Anhänger*innen finden
\end{itemize}
\section{Geister}
\begin{itemize} 
\item durch Duell tot 
\item Geister können nur mit anderen Geistern reden und ansonsten durch Briefnachrichten 

\chapter{Charakters} 
\section{Fürst M.}
Aufgabe: Du weißt um Ganjas falsche Charakter und um seine Heriatgsambitionen,
sowie das er eig. Aglaja liebt und willst (um beider willen) verhindern, dass
sie heiraten und sich somit beideseitig in den Abgrund stürzen.

\section{Rogoschin} 
Aufgabe: gehe zur R-Gruppe, überzeuge sie (mit Wein) ihm zu Folgen >>eine Party
zu crashen<<. Gehe mit der Gruppe zu Nastasja, Ganja, etc. und stelle Ganja bloß
(>>würde für 3 Rubel wie ein Hund über die Straße laufen<<), und biete Nastasja
30000 Rubel an. (Hinweis: du bist dir sich, du könntest bis zu 100 000 Rubel
auftreiben, sollte N.F. dein Angebot ablehen.

Hat zu beginn 30000 Rubel.

\section{Nastasja F.} 
Aufgabe: Du weißt die Gesellschaft trifft sich zu Beginn der Feier, um einen
>>Fürst Myschikin<< kennenzulernen. Eine ideale Gelegenheit, Ganja ein wenig zu
demütigen, die Aufmerksamkeit auf dich zu lenken.

direkt nach der Rege von XXX vor allen Gästen, verkündest du der gesammten
Gemeinschaft, ob u Ganja hereitest oder nicht. Dir kam dabei eine witzige Idee:
dieser Fürst M. (den du noch kennenlernen wirst) so ein bisschen ein Idiot sein.
Wäre es nicht funny, vor allen zu verkünden, er solle die Entscheidung treffen.

Trotz all deiner Wankelmütigkeiten, hast du dir fest vorgenommen, das die
Hochzeit, die an diesem stattfinden soll, \emph{deine} Hochzeit wird. Eine
Kandidat*in muss sich also finden.

Hinweis: Du nimmst kein anderen Heiratsantrag an, der dir nicht mindestens so
viel Geld bietet, wie der General und Tozki: 70 000

\section{Ipolit} 
Einladung R-Gruppe für >>erstes Kennenlernen<< (Hinweis:
Familienverplflichtungen, Raum für Pre-Spiel-Kommunikation, daher Auswahl
zwischen zwei Zugtickets)

\section{Kolja} 
Einladung R-Gruppe für >>erstes Kennenlernen<< (Hinweis:
Familienverplflichtungen, Raum für Pre-Spiel-Kommunikation, daher Auswahl
zwischen zwei Zugtickets)

Soll Rolle übernehmen, Nastasjas Hochzeit im letzten Moment durch ihren Weggang
zum Scheitern zu bringen. Entweder a) Charakterbeschreibung: Intuition, jede
Kanidat*in würde sie in den Abgrund stürzen, oder b) Anweisung von SL
(\emph{Überzeuge Nastasjas sobald sie vorm Altar steht dir zu folgen/ die
R-Gruppe zu treffen...})

\section{Der Idiot Charaktere}
\subsection{Erstes Treffen} 
\emph{Alle außer Rogoschin, und Nastasja:}\\
Vor kurzem Haben wir die Bekanntschafft unseres nahen verwandten Fürst
Myschikin gemacht. Es ergibt sich, dass er ebenfalls auf der Feier von XX
anwesend sein wird. Daher werden würden Ich mir wünschen euch zu sehen am YYY,
um ihn zu begrüßen. Er wird die nächsten Wochen bei den Iwolgins unterkommen. 

General Yepanchin.

\chapter{Random Ideas} 
Betrunken: blaue lippen (durch lebensmittelfarbe)

Wichtigsten (geklaute) Szenen aus Der Idiot, etc. werden vorgelesen (Tonband)

Ziel ihre Hochzeit zu feiern haben mehrere Charaktere: Ganja, Nastasja F.,
Gruschenka (Brüder K), ...

Nastasja F.~soll Hochzeit auf jeden Fall verlassen, entweder, wenn \emph{sie}
heiratet durch Kolja, oder falls jmd. anders heiratet durch SL (oder uns fällt
was besseres ein). Weggang führt zum Anschluss an die R-Gruppe.

\begin{itemize}
  \item[Wikipedia] >>In der Regel schildern die Schriftsteller […] nur solche Typen der
    Gesellschaft, die es in Wirklichkeit nur äußerst selten in so vollkommenen
    Exemplaren gibt, wie die Künstler sie darstellen, die aber als Typen
    nichtsdestoweniger fast noch wirklicher als die Wirklichkeit selbst sind.
    […] [I]n der Wirklichkeit [sei] das Typische der einzelnen Personen
    gewissermaßen wie mit Wasser verdünnt.<<
  \item[] Böse \emph{Geister} und \emph{Böse} Geister
  \item[] Charakterbeschreibungen Referenzieren Rogoschins 10000 Diebstahl. 
\end{itemize}

\end{itemize}
\end{document}
