\documentclass[12pt, a4paper, openany]{report}
\usepackage[left=3cm, top=3cm, bottom=3cm, right=4cm]{geometry}
%\DeclareUnicodeCharacter{00A0}{}
\usepackage[numbers]{natbib}
\usepackage{secdot} % Dots in Section Numbers
\usepackage[utf8]{inputenc}
\usepackage[T1]{fontenc}
\usepackage{siunitx}
\sisetup{math-micro=\text{µ},text-micro=µ}
%\usepackage[sfdefault]{arimo}
\usepackage{amsmath}
\usepackage{amssymb}
\usepackage{graphicx}
\usepackage{color}% http://ctan.org/pkg/color
\usepackage{hyperref}
\usepackage{verbatim}
\usepackage{acronym}
\usepackage{graphicx}
\usepackage{csquotes}
\usepackage{setspace}
\usepackage{tabulary}
\usepackage[table]{xcolor}
\usepackage[leftcaption]{sidecap} % <-- added
\usepackage{booktabs}
\usepackage{notoccite}
\definecolor{light-gray}{gray}{0.75}
\hyphenation{di-me-thyl-tryp-ta-mine}
\hyphenation{an-ti-de-pres-sant}
\hyphenation{psy-che-del-ics}
\hyphenation{pa-ra-form-al-de-hyd}
\usepackage[textwidth=90, backgroundcolor=blue]{todonotes}
\hypersetup{
	colorlinks=true,
	urlbordercolor={1 1 1},
	linkcolor=black,
	raiselinks=false,
	urlcolor=black,
	filecolor=black,
	citecolor=black,
	anchorcolor=black
}
\usepackage{fancyhdr}
\usepackage{titlesec}
\titleformat{\chapter}[hang]{\Huge\bfseries}{}{0pt}{\Huge\bfseries}

\pagestyle{fancy}
\fancyhf{}
\lhead{ananym \& fux}
\chead{Everyone is broken inside}
\rhead{\thepage}

\title{
  {\large{Dostojewski Larp}}\\
    {Everyone is broken inside}\\
    {\bigskip}
}
\author{ananym \& fux}
\date{\today}


\begin{document}
\maketitle 
\tableofcontents
\chapter{Einleitung}

\chapter{Gruppen}
\section{Kellerlöcher}
\begin{itemize}
\item Rätsel in der sie Stück für Stück Beziehungen von sich entdecken 
\item Rätsel Lösung sind Auschnitte aus Charaktermap
\item dadurch entsteht ein 2. Rätsel: Charakteramap zusammen puzzeln 
\item dadurch kriegen sie Hinweis für Revolutionäre Gruppe Nastasja zu entführen
\item Jeder Briefumschlag Geld, Bild (Foto) und Rätsel 
\item Lösung von Rästel ist ein Codewort das sie durch die Tür schieben zurück kommen drei Umschläge mit jeweils einem Namen (angebliche freundschaftliche Beziehungen) und je unterschiedlich manipulierter Ausschnitt aus der Charaktermap
\item eine Person SL schiebt neue Info durch Türschlitzt nachdem sie Lösung durch Türschlitz schieben 
\end{itemize}
 
\section{Die Reichen}
\begin{itemize}
\item kennen sich nicht, aber in Charakterbeschreibung, es gibt eine Gruppe die an \textbf{euer} Geld will. 
\item Daher wollen sie rausfinden wer diese Gruppe ist und deren Ziel verhindern
\end{itemize}

\section{Die Revolutionäre Gruppe}
\begin{itemize}
\item Ziel 1: Geld verbrennen 
\item Ziel 2: Anhänger*innen finden
\end{itemize}
\section{Geister}
\begin{itemize} 
\item durch Duell tot 
\item Geister können nur mit anderen Geistern reden und ansonsten durch Briefnachrichten 

\chapter{Random Ideas} 

\begin{itemize}
  \item[Wikipedia] >>In der Regel schildern die Schriftsteller […] nur solche Typen der
    Gesellschaft, die es in Wirklichkeit nur äußerst selten in so vollkommenen
    Exemplaren gibt, wie die Künstler sie darstellen, die aber als Typen
    nichtsdestoweniger fast noch wirklicher als die Wirklichkeit selbst sind.
    […] [I]n der Wirklichkeit [sei] das Typische der einzelnen Personen
    gewissermaßen wie mit Wasser verdünnt.<<
  \item[] Böse \emph{Geister} und \emph{Böse} Geister
  \item[] Charakterbeschreibungen Referenzieren Rogoschins 10000 Diebstahl. 
\end{itemize}

\end{itemize}
\end{document}
